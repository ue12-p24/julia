\documentclass{article}

\usepackage{xcolor}
\definecolor{jrouge}{HTML}{CB3C33}
\definecolor{jvert}{HTML}{389826}
\definecolor{jbleu}{HTML}{4063D8}
\definecolor{jviolet}{HTML}{9558B2}
\definecolor{lightred}{HTML}{fcf3f3}
\definecolor{lightgreen}{HTML}{e1f6db}
\definecolor{lightpurple}{HTML}{f4eef7}
\definecolor{lightgrey}{gray}{0.95}

\usepackage{natbib}
%\renewcommand{\citenumfont}[1]{{\tiny#1}}
\renewcommand{\citenumfont}[1]{}
\bibpunct{}{};s;;

\usepackage[top=1.5cm,bottom=1.5cm, left=2.5cm,right=2.5cm]{geometry}
\usepackage[french]{babel}
\usepackage[mathletters]{ucs}
\usepackage[utf8x]{inputenc}
\usepackage[T1]{fontenc}
% Or whatever. Note that the encoding and the font should match. If T1
% does not look nice, try deleting the line with the fontenc.
% Alternative for XeLaTeX:
%\usefonttheme{professionalfonts}
%\usepackage{fontspec}
%\setmonofont{JuliaMono}
%\setdefaultlanguage{french}
%\usepackage{unicode-math}

\usepackage{subcaption}

\usepackage{array}
\usepackage{multirow}
\usepackage{setspace}
\usepackage{soul}
\usepackage{amssymb}
\usepackage{mathrsfs}
\usepackage{bbm}
\usepackage{svg}

\usepackage{tikz}
\usetikzlibrary{scopes, backgrounds, arrows, automata, positioning, patterns, calc, decorations.pathmorphing, decorations.pathreplacing, arrows.meta}

\usepackage{hyperref}
\hypersetup{
	colorlinks=true,
	breaklinks=true,
}
\usepackage{ragged2e}
\usepackage{graphicx}
\usepackage{amsmath}
\usepackage{aeguill}

\usepackage{algorithm}
\usepackage{algpseudocode}
\usepackage{stmaryrd}
\usepackage{siunitx}

\usepackage{forloop}
\newcounter{loop}
\newcounter{numEx}
\newcommand{\exo}[1]{
	\stepcounter{numEx}
	\setcounter{loop}{0}
	\subsection*{Exercice \arabic{numEx} -- #1}
}
\newcommand{\correction}{\textsl{\\Correction de l'exercice \arabic{numEx}\\}}
\newenvironment{corr}{
	\correction	\begin{enumerate}}
	{\end{enumerate}
}

\usepackage{enumitem}
\renewlist{itemize}{itemize}{4}
\setlist[itemize,1]{label={$\bullet$}}
\setlist[itemize,2]{label={$\triangleright$}}
\setlist[itemize,4]{label={$\circ$}}

\newcommand{\llbra}{\left\llbracket}
\newcommand{\rrbra}{\right\rrbracket}
\renewcommand{\brack}[1]{\ensuremath{\llbra#1\rrbra}}
\newcommand{\der}[2]{#1^{\ensuremath{\left(#2\right)}}}
\newcommand{\paren}[1]{\ensuremath{\left(#1\right)}}
\newcommand{\abs}[1]{\ensuremath{\left|#1\right|}}
\newcommand{\interval}[1]{\ensuremath{\left[#1\right]}}
\newcommand{\set}[2]{\ensuremath{\left\{#1\,\middle|\,#2\right\}}}
\newcommand{\cont}[1]{\mathcal{C}^{#1}}
\newcommand{\tends}[2]{\underset{#1\to #2}{\longrightarrow}}
\newcommand{\seq}[3]{\ensuremath{\left(#1_{#2}\right)_{#2\in#3}}}
\newcommand{\matr}[2]{\mathcal{M}_{#1}\paren{#2}}
\newcommand{\matrRect}[3]{\mathcal{M}_{#1,#2}\paren{#3}}
\newcommand{\Id}{\text{Id}}
\newenvironment{disj}[1]{\left\{\begin{array}{#1}} {\end{array}\right.}
\newcommand{\N}{\mathbb{N}}
\newcommand{\R}{\mathbb{R}}
\newcommand{\C}{\mathbb{C}}
\newcommand{\Q}{\mathbb{Q}}
\newcommand{\Z}{\mathbb{Z}}
\newcommand{\K}{\mathbb{K}}
\newcommand{\eps}{\varepsilon}



\title{TD -- Apprentissage de la programmation en Julia}

\author{Lionel~Zoubritzky}

\date{11/2024}

\usepackage{minted}
\usemintedstyle{paraiso-light}
\setminted[julia]{mathescape,linenos,obeytabs,tabsize=4,numbersep=3pt,fontsize=\small,framesep=2mm,autogobble,bgcolor=lightred,escapeinside=££}
\setminted[bash]{mathescape,obeytabs,tabsize=4,numbersep=3pt,fontsize=\small,framesep=2mm,autogobble,bgcolor=lightgrey,escapeinside=££}

\usepackage{lmodern}
%\newcommand{\jl}[1]{\colorbox{lightred}{\small\ttfamily #1}}
\newmintinline[jl]{julia}{}
\newmintinline[jlscript]{julia}{fontsize=\scriptsize}
\newmint[JL]{julia}{}
\newmint[JLa]{julia}{linenos=false}
\newminted{julia}{}
\newenvironment{julia}{\vspace{-0.6em}\VerbatimEnvironment\begin{juliacode}}{\end{juliacode}}
\newminted[jlrepl]{julia}{linenos=false}
\newenvironment{repl}{\vspace{-0.6em}\VerbatimEnvironment\begin{jlrepl}}{\end{jlrepl}}
\newcommand{\q}{\textquotesingle}
\newcommand{\qq}{\textquotedbl}
\newcommand{\jlREPL}{\textcolor{jvert}{\bfseries julia>}}

\DeclareTextFontCommand{\emph}{\bfseries}

\usepackage{xspace}
\newcommand{\expr}{\ensuremath{\left\langle\textit{expr}\right\rangle}\xspace}
\newcommand{\expra}[1]{\ensuremath{\left\langle\textit{expr}_{#1}\right\rangle}\xspace}
\newcommand{\bexpr}{\ensuremath{\left\langle\textit{bexpr}\right\rangle}\xspace}
\newcommand{\bexpra}[1]{\ensuremath{\left\langle\textit{bexpr}_{#1}\right\rangle}\xspace}



\begin{document}
	
\begin{center}
	\Large Apprentissage de la programmation en Julia
	
	TD \no4 : Équations différentielles
	\vspace{2em}
\end{center}

Il n'est pas exigé que toutes les questions et tous les exercices soient traités. En particulier, les questions et exercices annotés du symbole $(*)$ ne sont à traiter qu'après avoir résolu les autres.

À chaque question pour laquelle il faut programmer une fonctionnalité, il est implicitement demandé d'inclure un ou plusieurs tests pour s'assurer que l'implémentation est correcte.

Toute fonction non-auxiliaire doit aussi inclure une docstring. Les commentaires sont partout bienvenus.\\

\textsl{Cette feuille de TD doit être réalisée dans un calepin électronique (Pluto ou Jupyter notebook).}


\exo{Dynamique proie-prédateur}


Les équations de Lotka-Volterra modélisent les populations de proie $x(t)$ et de prédateurs $y(t)$, étant donné quatre paramètres :
\begin{itemize}
	\item $\alpha$ le taux de reproduction des proies ;
	\item $\beta$ le taux de mortalité des proies dû à la prédation ;
	\item $\delta$ le taux de reproduction des prédateurs dû à l'abondance de gibier ;
	\item $\gamma$ le taux de mortalité intrinsèque des prédateurs.
\end{itemize}

Ces paramètres étant donnés la dynamique des population est donnée par :
\[\begin{disj}{rl}
	\text dx/\text dt &= x\times(\alpha - \beta y)\\
	\text dy/\text dt &= y\times(\delta x - \gamma)
\end{disj}\]

\begin{enumerate}
	\item Définir les valeurs des paramètres $\alpha$, $\beta$, $\gamma$ et $\delta$ de façon interactive entre 0 et 2. Faire de même pour la valeur $N_0$ de la population initiale, que l'on considérera égale pour les proies et les prédateurs entre 0.5 et 2.5.
	
	\item Implémenter le calcul de la solution de l'équation différentielle dans une fonction.

	\item $(*)$ Dessiner l'évolution de $x(t)$ et $y(t)$ au cours de la simulation.

	\item On suppose que $x(0) = y(0) = 2$, $\alpha = 2/3$, $\beta = 4/3$, $\gamma=1$ et $\delta=1$ et on résout le problème entre $t_{\min} = 0$ et $t_{\max} = 30$ avec l'intégrateur \jl{Tsit5()}.
	
	 en sauvegardant le résultat tous les $0.1$ unités de temps.
	\begin{enumerate}
		\item On utilise les arguments \jl{dt=5, saveat=5} dans la fonction \jl{solve}. Que valent $x(20)$ et $y(20)$ ?
		
		\item Même question en remplaçant \jl{5} par \jl{0.01}. Commenter.
	\end{enumerate}

	\item $(*)$ Dessiner le graphe des points de coordonnées $(x(t), y(t))$. On appelle ce graphe ``l'orbite'' de la solution.
	
	\item $(*)$ Expliquer le comportement limite de l'orbite dans le cas $\delta = 0$.

%	\item $(*)$ Pour $N_0 = 1$, y a-t-il une valeur des paramètres $\alpha$, $\beta$, $\delta$ et $\gamma$ dans la gamme prescrite telle que 
\end{enumerate}

\exo{Un peu d'épidémiologie}

On considère la propagation d'une épidémie au sein d'une populations de taille fixée à \jl{N} individus. Dans le modèle SIR, les individus ont trois états possibles :
\begin{itemize}
	\item S désigne les individus sains mais \textbf{S}usceptibles d'être infectés ;
	\item I désigne les individus \textbf{I}nfectés ;
	\item R désigne les individus \textbf{R}etirés, c'est-à-dire soit guéris et immunisés contre la maladie, soit morts.
\end{itemize}
La propagation de la maladie est liée à deux paramètres : le facteur de transmission $p$ et le temps moyen $\tau$ durant lequel un individu infecté est contagieux, de sorte que la dynamique des trois populations S, I et R est contrôlée par les équations :
\[\frac{\text dS}{\text dt} = - p\cdot I\cdot S\qquad\text{et}\qquad\frac{\text dR}{\text dt} = I/\tau\]


\begin{enumerate}
	\item Justifier que $\text dS/\text dt + \text dI/\text dt + \text dR/\text dt = 0$. En déduire l'équation qui régit l'évolution de $I$ en fonction de $R$ et $S$.

	\item Écrire une fonction \jl{simulate_epidemy(N, p, τ, Δt, S₀=N-1, R₀=0)} qui simule l'évolution des populations S, I et R dans ce modèle durant un temps \jl{Δt}, en supposant que les populations initiales sont données par $S(t=0) = \jl{S₀}$, $R(t=0) = \jl{R₀}$ et $I(t = 0) = \jl{N} - \jl{S₀} - \jl{R₀}$.

	\item $(*)$ Écrire une fonction qui dessine l'évolution dans le temps des populations S, I et R.
	
	\item $(*)$ Illustrer la dynamique des populations en fonction des valeurs de $p$ et $\tau$, choisies de façon interactive.

	\item Pour $p = \qty{2.73e-5}{h^{-1} individu^{-1}}$, $\tau = \qty{56.2}{h}$, $\jl{N} = \num{10000}$ et avec les valeurs par défaut de \jl{S₀}, \jl{I₀} et \jl{R₀}, quelle est la proportion d'individus encore S au bout de 2 jours ? Au bout d'une semaine ? Au bout d'un mois ?
	
	\item $(*)$ Pour ces mêmes valeurs de $p$ et $\tau$, quelle fraction de la population aurait-il fallu avoir vacciné avant le début de l'épidémie pour que la fraction maximale d'individus infectés ne dépasse pas \qty{20}\% ?
\end{enumerate}


%\section{Équations au dérivées partielles}
%
%\exo{Le soliton temporel}
%
%On considère l'équation de Korteweg--de Vries qui décrit l'évolution de la hauteur $\phi$ d'une vague scélérate :
%\[\frac{\partial\phi}{\partial t} + \frac{\partial^3\phi}{\partial x^3} - 6\phi\frac{\partial\phi}{\partial x} = 0\]

\exo{Cuisson d'une galette}

On modélise une galette par un pavé droit, de base carrée de côté $l = \qty{30}{cm}$ et de hauteur $h$. La galette est initialement à la température $T_{\text{init}} = \qty{300}K$.

La galette est plongée à l'instant $t = 0$ dans un four à la température $T_{\text{oven}}(t, z)$, que l'on modélise comme une simple température extérieure dépendant potentiellement du temps $t$ et la hauteur $z$, au contact de laquelle les trois surfaces de la galette se réchauffent\footnote{Un vrai four agit surtout par radiation mais on ignore cet aspect dans ce modèle}.

On rappelle l'équation de la chaleur :
\[\frac{\partial T}{\partial t} = \alpha\Delta T\]
avec $\alpha$ la diffusivité thermique, fixée à \qty{0.2}{mm^2/s} pour l'exercice, et $\Delta f = \frac{\partial^2f}{\partial x^2} + \frac{\partial^2f}{\partial y^2} + \frac{\partial^2f}{\partial z^2}$.

Pour simplifier le problème, on considère celui-ci symétrique selon les axes $x$ et $y$. On résout donc l'équation :
\[\frac{\partial T}{\partial t} = \alpha\times\paren{2\frac{\partial^2f}{\partial x^2} + \frac{\partial^2f}{\partial z^2}}\]

%On rappelle l'expression du laplacien en coordonnées cylindriques :
%\[\begin{split}\Delta f &= \frac1r\frac{\partial}{\partial r}\paren{r\frac{\partial f}{\partial r}} + \frac1{r^2}\frac{\partial^2f}{\partial\theta^2} + \frac{\partial^2f}{\partial z^2}
%\\&= \frac{\partial^2 f}{\partial r^2} + \frac1r\frac{\partial f}{\partial r} \hspace{0.33em}+ \frac1{r^2}\frac{\partial^2f}{\partial\theta^2} + \frac{\partial^2f}{\partial z^2}
%\end{split}\]

On fixe la température au bord de la galette comme valant
\[T_{\text{border}}(t, z) = T_{\text{init}} + \paren{T_{\text{oven}}(t, z)}\exp\paren{-t/\tau}\] avec $\tau = \qty{5}{minutes}$.

\begin{enumerate}
	\item Écrire une fonction qui renvoie les paramètres du problème, leurs domaines, la discrétisation spatiale, l'équation de la chaleur et les conditions au bord en fonction de $h$ et de la fonction $T_{\text{oven}}$, et qui renvoie un \jl{ODEProblem} (le résultat de \jl{discretize}).
	
	On simulera les 40 premières minutes de cuisson, on prendra 1 pas temporel par minute et 20 pas de discrétisation spatiale selon $(Ox)$ et selon $(Oz)$.
	
	\textit{Attention, les calculs peuvent demander un peu de temps, une vingtaine de secondes. La première exécution peut mettre plusieurs minutes.}
\end{enumerate}

On regarde la cuisson de la tarte \textbf{en oubliant la couche superficielle} : pour toutes les questions suivantes, il est demandé de ne pas prendre en compte les températures au bord de la galette, c'est-à-dire celles fixées par $T_{\text{border}}$.

Un élément de volume de la galette est cuit lorsqu'il a passé 20 minutes ou plus à \qty{140}{\degreeCelsius} ou plus. La galette est cuite lorsque chaque élément de volume est cuit.

Un élément de volume de la galette est brûlé lorsqu'il a passé 30 minutes ou plus à \qty{170}{\degreeCelsius} ou plus. La galette est brûlée s'il existe au moins un élément de volume brûlé.

\begin{enumerate}[resume]
	\item Écrire une fonction \jl{time_cooked} qui prend le résultat de la simulation et détermine le premier instant à partir duquel la galette est cuite.
	
	\item Faire de même qu'à la question précédente avec \jl{time_burnt}

	\item $(*)$ Afficher l'évolution de la température dans la galette en fonction de la position et du temps, sur une coupe de la galette qui passe par le centre de la galette.

	\item On fixe $h = \qty{3}{cm}$. On considère un four parfaitement homogène et préchauffé, de sorte que sa température est fixée à $T_{\text{oven}}(t, z) = T_0$ indépendamment du temps et de l'espace.
	\begin{enumerate}
		\item Si $T_0 = \qty{180}{\degreeCelsius}$, au bout de combien de temps la galette est-elle cuite ?
		\item Si $T_0 = \qty{200}{\degreeCelsius}$, au bout de combien de temps la galette est-elle brûlée ?
		\item $(*)$ Dans quelle gamme de température $T_0$ la galette peut-elle cuire en moins d'une heure sans brûler ?
	\end{enumerate}

	\item On fixe toujours $h = \qty{3}{cm}$, mais on considère maintenant un four initialement à température ambiante, et dont la répartition de la chaleur n'est pas homogène. La température du four est donnée par l'expression :
	\[T_{\text{oven}}(t, z) = \qty{300}K + \frac{\min(t, \qty{10}{minutes})}{\qty{10}{minutes}}\times\paren{T_0 - T_{\text{init}} - \qty{30}{\degreeCelsius}\times(1 - z/h)}\]
	
	Reprendre la question précédente dans ces conditions.
	
	\item $(*)$ Est-il plus facile de réussir sa galette dans le premier ou dans le second four ?
	
	\item $(*)$ Pour quelle gamme de hauteurs $h$ la galette peut-elle cuire en moins d'une heure sans brûler avec $T_0 = \qty{200}{\degreeCelsius}$, dans les deux types de four ?
	
%	\item $(*)$ On fixe $h =  \qty{3}{cm}$ et on reprend le four préchauffé à température uniforme $T_0$.
%	
%	Notre galette contient désormais une fève : celle-ci est modélisée par l'élément de volume située entre les hauteurs $h_{\text{inf}} = \qty{1.2}{cm}$ et $h_{\text{sup}} = \qty{1.5}{cm}$, entre les rayons $r_{\text{inf}} = \qty{4}{cm}$ et $r_{\text{sup}} = \qty{5.5}{cm}$ et entre les angles $\theta_{\text{inf}} = \qty{0}{\degree}$ et $\theta_{\text{sup}} = \qty{1}{\degree}$.
%	
%	Cette fève a pour diffusivité thermique $\alpha_{\text{fève}} = \qty{1.2}{mm^2/s}$ et on supposera que le bord de la fève est constamment à la température des éléments de volume de galette qui l'environnent.
%	\begin{enumerate}
%		\item Quelle est la température moyenne dans la fève au bout de 30 minutes de cuisson à $T_0 = \qty{200}{\degreeCelsius}$ ?
%		\item La galette met-elle plus ou moins de temps à cuire avec la fève ?
%	\end{enumerate}
\end{enumerate}

\end{document}