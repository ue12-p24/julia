\documentclass{article}

\usepackage{xcolor}
\definecolor{jrouge}{HTML}{CB3C33}
\definecolor{jvert}{HTML}{389826}
\definecolor{jbleu}{HTML}{4063D8}
\definecolor{jviolet}{HTML}{9558B2}
\definecolor{lightred}{HTML}{fcf3f3}
\definecolor{lightgreen}{HTML}{e1f6db}
\definecolor{lightpurple}{HTML}{f4eef7}
\definecolor{lightgrey}{gray}{0.95}

\usepackage{natbib}
%\renewcommand{\citenumfont}[1]{{\tiny#1}}
\renewcommand{\citenumfont}[1]{}
\bibpunct{}{};s;;

\usepackage[top=1.5cm,bottom=1.5cm, left=2.5cm,right=2.5cm]{geometry}
\usepackage[french]{babel}
\usepackage[mathletters]{ucs}
\usepackage[utf8x]{inputenc}
\usepackage[T1]{fontenc}
% Or whatever. Note that the encoding and the font should match. If T1
% does not look nice, try deleting the line with the fontenc.
% Alternative for XeLaTeX:
%\usefonttheme{professionalfonts}
%\usepackage{fontspec}
%\setmonofont{JuliaMono}
%\setdefaultlanguage{french}
%\usepackage{unicode-math}

\usepackage{subcaption}

\usepackage{array}
\usepackage{multirow}
\usepackage{setspace}
\usepackage{soul}
\usepackage{amssymb}
\usepackage{mathrsfs}
\usepackage{bbm}
\usepackage{svg}

\usepackage{tikz}
\usetikzlibrary{scopes, backgrounds, arrows, automata, positioning, patterns, calc, decorations.pathmorphing, decorations.pathreplacing, arrows.meta}

\usepackage{hyperref}
\hypersetup{
	colorlinks=true,
	breaklinks=true,
}
\usepackage{ragged2e}
\usepackage{graphicx}
\usepackage{amsmath}
\usepackage{aeguill}

\usepackage{algorithm}
\usepackage{algpseudocode}
\usepackage{stmaryrd}
\usepackage{siunitx}

\usepackage{forloop}
\newcounter{loop}
\newcounter{numEx}
\newcommand{\exo}[1]{
	\stepcounter{numEx}
	\setcounter{loop}{0}
	\subsection*{Exercice \arabic{numEx} -- #1}
}
\newcommand{\correction}{\textsl{\\Correction de l'exercice \arabic{numEx}\\}}
\newenvironment{corr}{
	\correction	\begin{enumerate}}
	{\end{enumerate}
}

\usepackage{enumitem}
\renewlist{itemize}{itemize}{4}
\setlist[itemize,1]{label={$\bullet$}}
\setlist[itemize,2]{label={$\triangleright$}}
\setlist[itemize,4]{label={$\circ$}}

\newcommand{\llbra}{\left\llbracket}
\newcommand{\rrbra}{\right\rrbracket}
\renewcommand{\brack}[1]{\ensuremath{\llbra#1\rrbra}}
\newcommand{\der}[2]{#1^{\ensuremath{\left(#2\right)}}}
\newcommand{\paren}[1]{\ensuremath{\left(#1\right)}}
\newcommand{\abs}[1]{\ensuremath{\left|#1\right|}}
\newcommand{\interval}[1]{\ensuremath{\left[#1\right]}}
\newcommand{\set}[2]{\ensuremath{\left\{#1\,\middle|\,#2\right\}}}
\newcommand{\cont}[1]{\mathcal{C}^{#1}}
\newcommand{\tends}[2]{\underset{#1\to #2}{\longrightarrow}}
\newcommand{\seq}[3]{\ensuremath{\left(#1_{#2}\right)_{#2\in#3}}}
\newcommand{\matr}[2]{\mathcal{M}_{#1}\paren{#2}}
\newcommand{\matrRect}[3]{\mathcal{M}_{#1,#2}\paren{#3}}
\newcommand{\Id}{\text{Id}}
\newenvironment{disj}[1]{\left\{\begin{array}{#1}} {\end{array}\right.}
\newcommand{\N}{\mathbb{N}}
\newcommand{\R}{\mathbb{R}}
\newcommand{\C}{\mathbb{C}}
\newcommand{\Q}{\mathbb{Q}}
\newcommand{\Z}{\mathbb{Z}}
\newcommand{\K}{\mathbb{K}}
\newcommand{\eps}{\varepsilon}



\title{TD -- Apprentissage de la programmation en Julia}

\author{Lionel~Zoubritzky}

\date{11/2024}

\usepackage{minted}
\usemintedstyle{paraiso-light}
\setminted[julia]{mathescape,linenos,obeytabs,tabsize=4,numbersep=3pt,fontsize=\small,framesep=2mm,autogobble,bgcolor=lightred,escapeinside=££}
\setminted[bash]{mathescape,obeytabs,tabsize=4,numbersep=3pt,fontsize=\small,framesep=2mm,autogobble,bgcolor=lightgrey,escapeinside=££}

\usepackage{lmodern}
%\newcommand{\jl}[1]{\colorbox{lightred}{\small\ttfamily #1}}
\newmintinline[jl]{julia}{}
\newmintinline[jlscript]{julia}{fontsize=\scriptsize}
\newmint[JL]{julia}{}
\newmint[JLa]{julia}{linenos=false}
\newminted{julia}{}
\newenvironment{julia}{\vspace{-0.6em}\VerbatimEnvironment\begin{juliacode}}{\end{juliacode}}
\newminted[jlrepl]{julia}{linenos=false}
\newenvironment{repl}{\vspace{-0.6em}\VerbatimEnvironment\begin{jlrepl}}{\end{jlrepl}}
\newcommand{\q}{\textquotesingle}
\newcommand{\qq}{\textquotedbl}
\newcommand{\jlREPL}{\textcolor{jvert}{\bfseries julia>}}

\DeclareTextFontCommand{\emph}{\bfseries}

\usepackage{xspace}
\newcommand{\expr}{\ensuremath{\left\langle\textit{expr}\right\rangle}\xspace}
\newcommand{\expra}[1]{\ensuremath{\left\langle\textit{expr}_{#1}\right\rangle}\xspace}
\newcommand{\bexpr}{\ensuremath{\left\langle\textit{bexpr}\right\rangle}\xspace}
\newcommand{\bexpra}[1]{\ensuremath{\left\langle\textit{bexpr}_{#1}\right\rangle}\xspace}



\begin{document}
	
\begin{center}
	\Large Apprentissage de la programmation en Julia
	
	TD \no3 : Dessin scientifique et bases de données
	\vspace{2em}
\end{center}

Il n'est pas exigé que toutes les questions et tous les exercices soient traités. En particulier, les questions et exercices annotés du symbole $(*)$ ne sont à traiter qu'après avoir résolu les autres.

À chaque question pour laquelle il faut programmer une fonctionnalité, il est implicitement demandé d'inclure un ou plusieurs tests pour s'assurer que l'implémentation est correcte.

Toute fonction non-auxiliaire doit aussi inclure une docstring. Les commentaires sont partout bienvenus.

\exo{Affichage d'une fonction}

On cherche à afficher la fonction $f:t\mapsto A\times\sin(\omega t) + B$ en fonction des paramètres $A$, $B$ et $\omega$.

\begin{enumerate}
	\item Créer un Pluto notebook dans lequel définir trois variables $A$, $B$ et $\omega$, liées à des curseurs de sélection sur $\brack{0, 10}$, $\brack{-5, 5}$ et $\brack{0, 20}$ respectivement.
	\item Afficher la fonction $f$ sur l'intervalle $t\in\left[-5, 5\right]$ et limiter les ordonnées à $\left[-6, 11\right]$.
	\item Trouver des valeurs de $A$, $B$ et $\omega$ telles que $f(-3) \le -1$, $3 < f(2) < 5$ et $f(4) \ge 5$.
\end{enumerate}

\exo{Un peu de mécanique du point}

Dans tout le problème, on suppose que le mouvement du système a lieu dans le plan $(Oxz)$ : on ne prendra donc en compte que deux dimensions.

On considère un système constitué d'une masse $m$ attachée au bout d'un fil élastique dont l'autre extrémité est fixée à l'origine du repère. Le fil est assimilé à un ressort de masse nulle, de longueur au repos $l_0$ et de constante de raideur $k$. L'ensemble est soumis à la gravitation terrestre, orientée vers les $z$ croissants.

\textsl{Cet exercice est à rendre de préférence dans un calepin électronique, comme un Pluto notebook.}

\begin{enumerate}
	\item Représenter le système par un type \jl{Pendulum} qui stocke les trois paramètres $m$, $l_0$ et $k$. Ceux-ci doivent pouvoir être donnés en unités adimensionnées, ou bien avec des unités de \texttt{Unitful.jl}.
	\item Écrire une fonction \jl{forces(::Pendulum, point)} qui calcule la résultante des forces appliquées sur la masse lorsqu'elle se situe aux coordonnées données en entrée.
	\item Écrire une fonction \jl{potentialenergy(::Pendulum, point)} qui calcule l'énergie potentielle du système à une position donnée.
	\item Écrire une fonction \jl{draw(::Pendulum, point)} qui dessine le pendule sous la forme d'un trait, allant de l'origine du repère au point donné, et dont la couleur va du rouge lorsque le ressort est contracté au bleu lorsqu'il est dilaté (en utilisant par exemple la palette de couleurs \jl{diverging_bkr_55_10_c35_n256}).
\end{enumerate}

On va simuler l'évolution temporelle d'un tel pendule élastique en intégrant les équations du mouvement de Newton. Pour ce faire, on découpe le temps en une suite d'instants, séparés par un pas de temps $\delta t$.

\begin{enumerate}[resume]
	\item En faisant l'approximation que pour toute fonction du temps $f(t)$ on a $f(t+\delta t) \approx f(t) + \delta tf'(t)$, en déduire un protocole de simulation de l'évolution temporelle du système précédent.
	\item Implémenter ce protocole en une fonction qui renvoie la listes des paires formées par la position et la vitesse de la masse à chaque pas de la simulation. La signature de la fonction est :
	
	\jl{euler_integration(::Pendulum, initial_point, initial_velocity, δt, total_time)}.
	\item En utilisant $k = \qty{100}{J/m^2}$, $l_0 = \qty{20}{cm}$, $m = \qty{1}{kg}$, $\delta t = \qty{1}{ms}$, une vitesse initiale nulle, et pour point de départ $(-l_0, 0)$, dessiner l'évolution de la coordonnée $x$ du système au cours du temps pendant une minute simulée.
	\item Dessiner l'évolution de l'énergie totale du système au cours du temps. Qu'observe-t-on ?
\end{enumerate}

On utilise plutôt un schéma d'intégration qui s'appelle \textit{velocity Verlet}. Cela consiste à utiliser l'approximation suivante, en notant $\textbf x(t) = f(t)$, $\textbf v(t) = f'(t)$ et $\textbf a(t) = f''(t)$ :
\[\begin{split}
	\textbf x(t + \delta t) &\approx \textbf x(t) + \textbf v(t)\delta t + \frac12 \textbf a(t)\delta t^2\\
	\textbf v(t + \delta t) &\approx \textbf v(t) + \frac{\textbf a(t) + \textbf a(t+\delta t)}2\delta t
\end{split}\]

\begin{enumerate}[resume]
	\item $(*)$ En utilisant des développements de Taylor, calculer l'erreur sur $f(t+\delta t)$ exprimée en grand $O$ d'une puissance de $\delta t$. Comparer au schéma d'intégration d'Euler (de la question 4).
	\item Implémenter ce nouveau schéma d'intégration sous la forme d'une fonction \jl{verlet_integration} prenant les mêmes paramètres que \jl{euler_integration}. Reprendre ensuite les questions 7 et 8 en utilisant cette fois-ci \jl{verlet_integration}.
	\item Superposer sur un même dessin l'évolution de la coordonnée $x$ en fonction du temps pour différentes valeurs de $k$, allant de \qty{100}{J/m^2} à \qty{150}{J/m^2}. Commenter.
	\item $(*)$ Écrire une fonction \jl{draw_simulation(::Pendulum, output)} qui prend le pendule ainsi que la valeur renvoyée par \jl{verlet_integration} comme entrée, et crée une animation montrant visuellement la trajectoire du système au cours du temps. On reprendra la même représentation du pendule qu'à la question 4.
\end{enumerate}

On considère maintenant plusieurs pendules élastiques comme celui précédent, attachés les uns aux autres, le premier étant toujours attaché à l'origine du repère.

\begin{enumerate}[resume]
	\item $(*)$ Représenter ce nouveau système par un type \jl{PendulumSequence} dont un unique champ contient une liste de \jl{Pendulum}s ayant tous le même type mais pas nécessairement les mêmes valeurs des paramètres $m$, $l_0$ et $k$.
	\item $(*)$ Reprendre les questions 2 et 3 pour ce nouveau type \jl{PendulumSequence}. Le second paramètre \jl{point} doit être une liste de points, un pour chaque sous-pendule.
	\item $(*)$ On considère le cas de deux pendules élastiques attachés ensemble, le premier de paramètres $\der k1 = \qty{100}{J/m^2}$, $\der{l_0}1 = \qty{20}{cm}$, $\der m1 = \qty{1}{kg}$, et le second avec $\der k2 = \qty{50}{J/m^2}$, $\der{l_0}2 = \qty{10}{cm}$, $\der m2 = \qty{0.1}{kg}$. Les deux masses sont initialement disposées en respectivement $(-\der{l_0}1, 0)$ et $(-\der{l_0}1-\der{l_0}2, 0)$, lâchées sans vitesse initiale.

	Combien de temps met ce système à atteindre une configuration où la seconde masse se retrouve au-dessus de l'axe $(Ox)$ ?
	\item $(*)$ Reprendre la question 12 avec le nouveau type \jl{PendulumSequence}.
\end{enumerate}


\end{document}